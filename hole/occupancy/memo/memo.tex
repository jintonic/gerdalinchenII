\documentclass[12pt, a4paper]{article}
\input{preamble}

\title{Occupancy analysis using 88 keV line from Cd source in the core
  of Siegfried II}

\author{Jing Liu}

\date{\today{}}

\begin{document}
\maketitle{}
\tableofcontents

\section{Data selection}
\label{s:sel}
Data used here were taken in Run\_05 with Siegfried~II (SII in short)
in Gerdalinchen~II (GII in short):
\begin{lstlisting}
/.hb/raidg01/Siegfried_II/Run_05/Cd_ET/Cd_et000{1,4,6}.root
\end{lstlisting}
The selection codes are in:
\begin{lstlisting}
  ~/work/gerdalinchenII/hole/occupancy/data/SII_Run_05_Cd_ET.C
\end{lstlisting}
The selection is done in several steps:
\begin{enumerate}
\item \lstinline!void pre(Int_t index)! selects single segment event
  (energy threshold is 15~keV) with fired segment only in the top
  layer of SII and the energy deposited is smaller than 100~keV. The
  output are pre1.root, pre4.root and pre6.root.
\item \lstinline!void mca(Int_t index)! reads pre1.root, pre4.root or
  pre6.root, and create the MCA spectra of all channels (mca[0] means
  the core) in the range of about [70, 100]~keV. The output are
  mca1.root, mca4.root and mca6.root.
\item \lstinline!void fitmca(Int_t index)! reads mca1.root, mca4.root
  or mca6.root, gets the MCA spectra of segment 1, 2, 3, 16, 17, 18
  (top layer), relabels them to 1, 2, 3, 4, 5, 6 as shown in Fig.~
  \ref{f:setup}. A Gaussian plus $1^{st}$-order polynomial is used to
  fit the 88~keV peak in each of the spectrum as shown in
  Fig.~\ref{f:fit88}. The numbers of events inside the peak and the
  relabeled segment ID are output into Nseg1.txt, Nseg4.txt and
  Nseg6.txt.
\end{enumerate}

\begin{figure}[fptb]
  \centering
  \includegraphics[width=0.8\textwidth]{segID}
  \caption{Experimental setup.}
  \label{f:setup}
\end{figure}

\begin{figure}[fptb]
  \centering
  \includegraphics[width=\textwidth]{fit88}
  \caption{Fit of the 88~keV peaks recorded by the segments in the top
    layer of SII.}
  \label{f:fit88}
\end{figure}
The codes can be run as:
\begin{lstlisting}
  $ cd ~/work/gerdalinchenII/hole/occupancy/data/
  $ root
  root[] .L SII_Run_05_Cd_ET.C++
  root[] pre(1) or pre(4), pre(6)
  root[] mca(1) or mca(4), mca(6)
  root[] fitmca(1) or fitmca(4), fitmca(6)
\end{lstlisting}

\section{Simulation}
\label{s:sim}
The geometry is described in file 
\begin{lstlisting}
  ~/work/gerdalinchenII/hole/occupancy/sim/SII_Cd.gdml
\end{lstlisting}
The deadlayer in the core is 0.5~mm thick.

The hits distribution is simulated using MaGe. The macro file is:
\begin{lstlisting}
  ~/work/gerdalinchenII/hole/occupancy/sim/SII_Cd.mac
\end{lstlisting}
The source is a small tablet emitting 88~keV photons in the core of
SII. 200,000 events are simulated. The output file is
\lstinline!SII_Cd.root!. The R distribution of the hits is shown in Fig.~\ref{f:dep}.
\begin{figure}[fptb]
  \centering
  \includegraphics[width=0.5\textwidth]{Depth88keV}
  \caption{R distribution of hits created by the $^109$Cd source in
    the core of SII.}
  \label{f:dep}
\end{figure}

The drift of holes created by the hits is calculated using the program
\begin{lstlisting}
  ~/work/gerdalinchenII/hole/occupancy/sim/calculatePhis.cc
\end{lstlisting}
The program is linked to \lstinline!MaGe/sandbox!, compiled there.

The positions of the end points of the drift trajectories on the outer
surface of SII are recorded in files \lstinline!Occ*.root!.

The electric field is calculated using program
\lstinline!calculateFields.cc!. It changes with the impurity density
inside the crystal. Different impurity density distributions are
tried:
\begin{itemize}
\item zero impurity. the field is saved in
  \lstinline!f0impurity.root!.
\item $4.1 \times 10^{9}/$cm$^{3}$. the field is saved in
  \lstinline!f4e9.root!.
\item $1 \times 10^{10}/$cm$^{3}$. the field is saved in
  \lstinline!f1e10.root!.
\end{itemize}

\section{Comparison}
\label{s:com}
The comparison is done using the function 
\lstinline!void occupancy(Int_t index=1)! in program:
\begin{lstlisting}
  ~/work/gerdalinchenII/hole/occupancy/data/SII_Run_05_Cd_ET.C
\end{lstlisting}
The input include
\begin{lstlisting}
  ~/work/gerdalinchenII/hole/occupancy/data/Nseg1{4,6}.txt
\end{lstlisting}
which contains the occupancy distribution in data, and
\begin{lstlisting}
  ~/work/gerdalinchenII/hole/occupancy/sim/Occ*.root
\end{lstlisting}
which contains the information of the simulated end points of hole
trajectories on the outer surface of SII. The signals are assigned to
segment 1, 2, 3, 4, 5, 6 according to the $\phi$ angle of the end
points. The simuated occupancy changes with $\phi_{110}$, the angle
between the crystal axis and one segment boundary. In function
\lstinline!void occupancy(Int_t index=1)!, $\phi_{110}$ changes from
0$^{\circ}$ to 360$^{\circ}$ with a step size of 1$^{\circ}$. For each
$\phi_{110}$ value a occupancy distribution is calculated and compared
to data after normalization. The $\chi^{2}$ of the comparison is
calculated as:
\begin{equation}
  \label{e:chi2}
  \chi^{2} = \sum_{i=1}^{6} \left( \frac{N^{dat}_{i} - N^{nor}_{i}} 
    {\sigma_{i}} \right)^{2} \text{   with   } N^{nor}_{i} = 
  N^{sim}_{i} \frac{\sum N^{dat}_{i}} {\sum N^{sim}_{i}} ,
\end{equation}
where $i$ indicates the segment ID as defined in Fig.~\ref{f:setup};
$N^{dat}_{i}$ and $N^{sim}_{i}$ indicate the number of 88~keV events
in segment $i$ given by the measurement and simulation, respectively;
and $\sigma_{i}$ is the total uncertainty from the measurement and
simulation which can be expressed as
\begin{equation}
  \label{e:sig}
  \sigma_{i}^{2} = (\sigma_{i}^{dat})^{2} + (\sigma_{i}^{nor})^{2}
\end{equation}
with
\begin{eqnarray}
  \label{e:nor}
  (\frac{\sigma_{i}^{nor}}{N_{i}^{nor}})^{2} &=& 
  (\frac{\sigma_{i}^{sim}}{N_{i}^{sim}})^{2} + 
  (\frac{\sigma_{i}^{\sum dat}}{\sum N_{i}^{dat}})^{2} +
  (\frac{\sigma_{i}^{\sum sim}}{\sum N_{i}^{sim}})^{2}\\
  &=& \frac{1}{N_{i}^{sim}} + \frac{1}{\sum N_{i}^{dat}} +
  \frac{1}{\sum N_{i}^{sim}}.
\end{eqnarray}
So the $\chi^{2}$ can be expressed as
\begin{equation}
  \label{e:fx2}
  \chi^{2} = \sum_{i=1}^{6} \frac{(N_{i}^{dat} - N_{i}^{nor})^{2}} 
  {(\sigma_{i}^{dat})^{2} + (N_{i}^{nor})^{2}(\frac{1}{N_{i}^{sim}} + 
    \frac{1}{\sum N_{i}^{dat}} + \frac{1}{\sum N_{i}^{sim}})}
\end{equation}
The number of degrees of freedom (NDF) of the fit is
\begin{equation}
  \label{e:ndf}
  \text{NDF} = 6 \text{ (data points)} - 1 \text{ (free parameter } \phi_{110}) = 5
\end{equation}

The $\chi^{2}$/NDF as a function of $\phi_{110}$ is shown in the right
plot of Fig.~\ref{f:occ4e9}. Different impurity densities in the
simulation and different data sets are tried, and $\phi_{110}$ with
the smallest $\chi^{2}$ given by the fits is listed in
Table~\ref{t:res}. 

\begin{figure}[pbht]
  \centering
  \includegraphics[width=0.9\textwidth]{occ0_1}
  \includegraphics[width=0.9\textwidth]{occ4e9_1}
  \includegraphics[width=0.9\textwidth]{occ1e10_1}
  \caption{Left: Occupancy distribution from data and simulation with
    the smallest $\chi^2$, Right: $\chi^2$ as a function of
    $\phi_{110}$. The impurity density in the simulation is 0, 0.41
    and 1 $\times 10^{10}$~cm$^{-3}$ from top to bottom.}
  \label{f:occ4e9}
\end{figure}

\begin{table}[pbht]
  \centering
  \caption{$\phi_{110}$ corresponding to the minimal $\chi^{2}$/NDF 
    (in the brackets).}
  \label{t:res}
\begin{tabular}{c|c|c|c}
\hline
impurity ($10^{10}$~cm$^{-3}$) & 0 & 0.41 (Canberra value)& 1 \\\hline
data set 1 & 174$^\circ$ (33) & 174$^\circ$ (38) & 224$^\circ$ (234) \\
data set 4 & 175$^\circ$ (37) & 174$^\circ$ (42) & 226$^\circ$ (250) \\
data set 6 & 176$^\circ$ (35) & 175$^\circ$ (41) & 226$^\circ$ (242) \\\hline
\end{tabular}
\end{table}

The value of $\phi_{110}$ does change with the impurity density. The
smaller the density, the stronger the bend of the trajectories, and
the larger the difference of the occupancy in different segments. When
the impurity density in the simulation is too large, the difference of
the occupancy in different segments becomes too small to describe the
data. It gives us a feeling of the impurity level. Once the impurity
density is within a reasonable region, the result does not change too
much.

The best results are given by the simulation with zero impurity
instead of the value given by Canberra. This is understandable because
the value given by Canberra is a average, the top layer should have a
smaller value.

Different data sets give slightly different results. The deviation is
about 2$^{\circ}$. Since the $\chi^{2}$/NDF distribution has a
$90^{\circ}$ degeneracy, $\phi_{110}$ and $\phi_{110} + n \times
90^{\circ}$ (n is an integer) gives the same result, the fit indicates
that $\phi_{110} \cong -5^{\circ}$.

\end{document}

%%% Local Variables:
%%% mode: latex
%%% TeX-master: t
%%% End: 
