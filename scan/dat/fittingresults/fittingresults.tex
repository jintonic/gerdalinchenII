%--------------------------------------------------------
% style 
%--------------------------------------------------------

\documentclass[11pt, a4paper]{article}

%\usepackage{epsfig}
\usepackage{subfigure}
\usepackage[dvips]{graphicx}
\usepackage{makeidx}
\usepackage{multirow}
\usepackage{verbatim}
\setlength{\oddsidemargin}{0cm}
\setlength{\evensidemargin}{0cm}
\setlength{\topmargin}{-1cm}
\setlength{\textheight}{23cm}
\setlength{\textwidth}{16cm}


\newcommand{\mage}     {{\sc MaGe}}
\newcommand{\geant}    {{\sc GEANT4}}
\newcommand{\rootv}    {{\sc ROOT}}
\newcommand{\gerda}    {{\sc GERDA}}
\newcommand{\majorana} {{\sc Majorana}} 
\newcommand{\decay}   {{\sc DECAY0}}
\newcommand{\eqref}[1]{Eq.\,(\ref{#1})}
\newcommand{\eqsref}[2]{Eqs.\,(\ref{#1}),\,(\ref{#2})}
\newcommand{\figref}[1]{Fig.\,\ref{#1}}
\newcommand{\figsref}[2]{Figs.\,\ref{#1},\,\ref{#2}}
\newcommand{\cms}{c.m.\,}
 
%--------------------------------------------------------

\begin{document}
\begin{titlepage}



\hspace{10.8cm} March 2009  

\begin{center}
{\Large Localization of the SiegrfriedII segments by means of the characteristic 121keV Europium peak }

\vspace{9.0cm}

\begin{abstract}
The Gerdalinchen experiment at the "Max Planck Institut f\"ur Physik M\"unchen" makes use of segmented Germanium detectors to prepare the search for the neutrinoless double $\beta$-decay at LNGS. Cylindric detectors are therefore divided along their phi- and z-axes into 6 $\cdot$ 3 parts. In order to localize the position of the segments in the whole setup a Europium source was moved around the detector. Here the results of the measurement along the phi-axes are reported.
\end{abstract}


\end{center}
\end{titlepage}


% -------------------------------------------------------- 

\begin{table}[h]
\begin{center}
\begin{tabular}{|c|c|c|c|c|c|c|c|c|}
\hline
angle	&$N_{core}$	&$\sigma_{core}$	&$N_{13}$	&$\sigma_{13}$		&$N_{14}$	&$\sigma_{14}$		&$N_{15}$	&$\sigma_{15}$	&\hline
5     &406 $\pm$ 60    &30			&x   $\pm$ x 	 &752			&x   $\pm$ x	&457			&552 $\pm$ 105		&150	& \hline
10    &438 $\pm$ 60    &43			&136 $\pm$ 101   &172   		&12  $\pm$ 74   &14 			&581 $\pm$ 119		&137	& \hline
15    &390 $\pm$ 62    &41			&148 $\pm$ 239   &192   		&164 $\pm$ 127  &173			&145 $\pm$ 70		&90	& \hline
20    &321 $\pm$ 62    &38			&-25 $\pm$ 683   &2			&482 $\pm$ 155  &164			&175 $\pm$ 132		&176	& \hline
25    &336 $\pm$ 60    &39			&62  $\pm$ 55    &83			&307 $\pm$ 58   &77			&147 $\pm$ 90		&171	& \hline
30    &445 $\pm$ 71    &38			&52  $\pm$ 23    &22			&436 $\pm$ 42   &47			&65  $\pm$ 29		&37	& \hline
35    &285 $\pm$ 47    &24			&32  $\pm$ 207   &3			&466 $\pm$ 43   &52			&29  $\pm$ 22		&19	& \hline
40    &490 $\pm$ 74    &46			&22  $\pm$ 12    &5			&423 $\pm$ 40   &44			&33  $\pm$ 17		&19	& \hline
45    &446 $\pm$ 76    &47			&224 $\pm$ 113   &167			&488 $\pm$ 43   &47			&22  $\pm$ 179		&29	& \hline
50    &395 $\pm$ 59    &35			&x   $\pm$ x     &819			&515 $\pm$ 39   &43			&30  $\pm$ 14		&15	& \hline
55    &442 $\pm$ 61    &34			&50  $\pm$ 308   &2			&572 $\pm$ 43   &51			&0   $\pm$ 3		&0	& \hline
60    &452 $\pm$ 66    &38			&51  $\pm$ 20    &17			&317 $\pm$ 71   &104			&x   $\pm$ x		&x	& \hline
65    &602 $\pm$ 69    &49			&-19 $\pm$ 12    &7			&423 $\pm$ 84   &129			&16  $\pm$ 30		&8	& \hline
70    &529 $\pm$ 66    &45			&30  $\pm$ 26    &16			&445 $\pm$ 107  &146			&2   $\pm$ 477		&35	& \hline
75    &350 $\pm$ 44    &36			&158 $\pm$ 38    &44			&62  $\pm$ 87   &43			&17  $\pm$ 7		&8	& \hline
80    &454 $\pm$ 54    &34			&422 $\pm$ 97    &124			&-16 $\pm$ 14   &10			&15  $\pm$ 110		&20	& \hline
85    &663 $\pm$ 73    &49			&508 $\pm$ 76    &101			&-108$\pm$ 55   &93			&46  $\pm$ 107		&36	& \hline
90    &425 $\pm$ 58    &38			&633 $\pm$ 197   &183			&32  $\pm$ 19   &21			&33  $\pm$ 43 		&47	& \hline










\end{tabular}
\end{center}
\caption{results1}
\label{tab:ranger}
\end{table}

\end{document}

\end{document}
