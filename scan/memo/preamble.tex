\usepackage[reals]{layout} %provide \layout macro to show current page layout

\usepackage{cite}

\usepackage{listings}
\lstset{ 
  language=C++,
  morekeywords={Float_t, Int_t, Double_t, Bool_t},
  frame=tbrl,
  frameround=tttt,
  showspaces=false,
  showtabs=false,
  basicstyle=\footnotesize,
% backgroundcolor=\color{Gray},
% fillcolor=\color{Gray},
  extendedchars=true
}            
\lstloadlanguages{sh,bash,csh,[GNU]C++,[gnu]make,SQL}

\usepackage{graphicx}
\usepackage{subfigure}
\usepackage{wrapfig}

\usepackage{amsmath}            % more evironment
\usepackage{amssymb}            % more symbol

\usepackage{ifpdf}
\ifpdf
\usepackage{epstopdf} % must be put after graphicx
\usepackage[usenames,dvipsnames]{color}
\usepackage[pdftex,bookmarks=true]{hyperref}
\hypersetup{
  pdfauthor = {Jing Liu},
  pdftitle = {memo for scanning of SII},
  pdfkeywords = {pulse shape simulation},
}
\pdfadjustspacing=1 %force pdfLaTeX to use the same spacing as LaTeX
\else
\usepackage[usenames,dvips]{color}
\usepackage[ps2pdf]{hyperref}
\fi

\setlength{\oddsidemargin}{0.8cm}
\setlength{\evensidemargin}{0cm}
\setlength{\textwidth}{15cm}
\setlength{\textheight}{21cm}
\setlength{\hoffset}{0cm}
\setlength{\voffset}{0cm}

% Alter some LaTeX defaults for better treatment of floats: See p.105 % of "TeX Unbound" for suggested values. See pp. 199-200 of Lamport's % "LaTeX" book for details.

% General parameters, for ALL pages:
\renewcommand{\topfraction}{0.9} % max fraction of floats at top
\renewcommand{\bottomfraction}{0.9} % max fraction of floats at bottom

% Parameters for TEXT pages (not float pages):
\setcounter{topnumber}{2}
\setcounter{bottomnumber}{2}
\setcounter{totalnumber}{4} % 2 may work better
\renewcommand{\textfraction}{0.1} % allow minimal text w. figs

% Parameters for FLOAT pages (not text pages):
\renewcommand{\floatpagefraction}{0.7}	% require fuller float pages
% N.B.: floatpagefraction MUST be less than topfraction !!

% remember to use [htp] or [htpb] for placement
%\pagestyle{headings}

%% The lineno packages adds line numbers. Start line numbering with
%% \begin{linenumbers}, end it with \end{linenumbers}. Or switch it on
%% for the whole article with \linenumbers.
\usepackage{lineno}

%%% Local Variables: 
%%% mode: latex
%%% TeX-master: "memo"
%%% End: 
